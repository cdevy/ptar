\documentclass[french,12pt,a4paper,titlepage]{article}
\usepackage[utf8]{inputenc}
\usepackage[T1]{fontenc}
\usepackage{babel}
\usepackage{hyperref}
\usepackage{graphicx}
\usepackage[table]{xcolor}
\usepackage{lastpage}
\usepackage{fancyhdr}
\pagestyle{fancy}
\renewcommand\headrulewidth{1pt}
\fancyhead[L]{Rapport de projet de RS}
\renewcommand\footrulewidth{1pt}
\fancyfoot[L]{TELECOM Nancy 2A}
\fancyfoot[C]{\textbf{Page \thepage/\pageref{LastPage}}}
\fancyfoot[R]{Année 2016-2017}
\title{%
  {\Huge \scshape Module de Réseaux-Systèmes}\\
  \vspace{2.5cm}
  {\Huge \textbf{Rapport de projet}}\\
  {\emph{Sujet} : ptar – un extracteur d’archives tar durable et
parallèle}\\ \vspace{2.5cm}}
\author{Charlotte {\scshape Devy}\\Guillaume {\scshape Ruchot}}
\date{Date de rendu : 16 décembre 2016}

\begin{document}

\begin{titlepage}
	\begin{figure}[t]
		\includegraphics[scale=0.3]{Telecom_nancy.png}
	\end{figure}
\end{titlepage}
\maketitle

\tableofcontents

\newpage

\section*{Remerciements} \addcontentsline{toc}{section}{Remerciements} \markboth{\scshape \large remerciements}{}
\paragraph{} $ < $ Votre rapport doit contenir quelque chose comme la mention suivante : "Nous avons réalisé ce projet sans aucune forme d’aide extérieure" ou bien "Nous avons réalisé ce projet en nous aidant des sites webs suivants (avec une liste de sites, et les informations que vous avez obtenues de chaque source)" ou encore "Nous avons réalisé ce projet avec l’aide de (nommez les personnes qui vous ont aidé, et indiquez ce qu’ils ont fait pour vous)". $ > $

\newpage

\section{Conception}

\newpage

\section{Réalisation}

\newpage

\section{Gestion de projet}
\paragraph{} Ci-dessous, le tableaux récapitulatif du nombre d'heures passées sur chaque partie du projet par chacun des membres :\\
\newline
\newline
\rowcolors{2}{white}{blue!10}
\begin{tabular}[c]{lcc}
	&	\textbf{Charlotte DEVY}	&	\textbf{Guillaume RUCHOT}	\\
\textbf{Conception}	&	h	&	h	\\
\textbf{Codage}	&	h	&	h	\\
\textbf{Tests}	&	h	&	h	\\
\textbf{Rédaction du rapport}	&	h	&	h	\\
\textbf{\scshape Total}	&	h	&	h	\\
\end{tabular}

\newpage

\end{document}
